% Options for packages loaded elsewhere
\PassOptionsToPackage{unicode}{hyperref}
\PassOptionsToPackage{hyphens}{url}
%
\documentclass[
]{book}
\usepackage{lmodern}
\usepackage{amssymb,amsmath}
\usepackage{ifxetex,ifluatex}
\ifnum 0\ifxetex 1\fi\ifluatex 1\fi=0 % if pdftex
  \usepackage[T1]{fontenc}
  \usepackage[utf8]{inputenc}
  \usepackage{textcomp} % provide euro and other symbols
\else % if luatex or xetex
  \usepackage{unicode-math}
  \defaultfontfeatures{Scale=MatchLowercase}
  \defaultfontfeatures[\rmfamily]{Ligatures=TeX,Scale=1}
\fi
% Use upquote if available, for straight quotes in verbatim environments
\IfFileExists{upquote.sty}{\usepackage{upquote}}{}
\IfFileExists{microtype.sty}{% use microtype if available
  \usepackage[]{microtype}
  \UseMicrotypeSet[protrusion]{basicmath} % disable protrusion for tt fonts
}{}
\makeatletter
\@ifundefined{KOMAClassName}{% if non-KOMA class
  \IfFileExists{parskip.sty}{%
    \usepackage{parskip}
  }{% else
    \setlength{\parindent}{0pt}
    \setlength{\parskip}{6pt plus 2pt minus 1pt}}
}{% if KOMA class
  \KOMAoptions{parskip=half}}
\makeatother
\usepackage{xcolor}
\IfFileExists{xurl.sty}{\usepackage{xurl}}{} % add URL line breaks if available
\IfFileExists{bookmark.sty}{\usepackage{bookmark}}{\usepackage{hyperref}}
\hypersetup{
  pdftitle={Yet Another Trading Algorithm},
  pdfauthor={Grandez},
  hidelinks,
  pdfcreator={LaTeX via pandoc}}
\urlstyle{same} % disable monospaced font for URLs
\usepackage{color}
\usepackage{fancyvrb}
\newcommand{\VerbBar}{|}
\newcommand{\VERB}{\Verb[commandchars=\\\{\}]}
\DefineVerbatimEnvironment{Highlighting}{Verbatim}{commandchars=\\\{\}}
% Add ',fontsize=\small' for more characters per line
\usepackage{framed}
\definecolor{shadecolor}{RGB}{248,248,248}
\newenvironment{Shaded}{\begin{snugshade}}{\end{snugshade}}
\newcommand{\AlertTok}[1]{\textcolor[rgb]{0.94,0.16,0.16}{#1}}
\newcommand{\AnnotationTok}[1]{\textcolor[rgb]{0.56,0.35,0.01}{\textbf{\textit{#1}}}}
\newcommand{\AttributeTok}[1]{\textcolor[rgb]{0.77,0.63,0.00}{#1}}
\newcommand{\BaseNTok}[1]{\textcolor[rgb]{0.00,0.00,0.81}{#1}}
\newcommand{\BuiltInTok}[1]{#1}
\newcommand{\CharTok}[1]{\textcolor[rgb]{0.31,0.60,0.02}{#1}}
\newcommand{\CommentTok}[1]{\textcolor[rgb]{0.56,0.35,0.01}{\textit{#1}}}
\newcommand{\CommentVarTok}[1]{\textcolor[rgb]{0.56,0.35,0.01}{\textbf{\textit{#1}}}}
\newcommand{\ConstantTok}[1]{\textcolor[rgb]{0.00,0.00,0.00}{#1}}
\newcommand{\ControlFlowTok}[1]{\textcolor[rgb]{0.13,0.29,0.53}{\textbf{#1}}}
\newcommand{\DataTypeTok}[1]{\textcolor[rgb]{0.13,0.29,0.53}{#1}}
\newcommand{\DecValTok}[1]{\textcolor[rgb]{0.00,0.00,0.81}{#1}}
\newcommand{\DocumentationTok}[1]{\textcolor[rgb]{0.56,0.35,0.01}{\textbf{\textit{#1}}}}
\newcommand{\ErrorTok}[1]{\textcolor[rgb]{0.64,0.00,0.00}{\textbf{#1}}}
\newcommand{\ExtensionTok}[1]{#1}
\newcommand{\FloatTok}[1]{\textcolor[rgb]{0.00,0.00,0.81}{#1}}
\newcommand{\FunctionTok}[1]{\textcolor[rgb]{0.00,0.00,0.00}{#1}}
\newcommand{\ImportTok}[1]{#1}
\newcommand{\InformationTok}[1]{\textcolor[rgb]{0.56,0.35,0.01}{\textbf{\textit{#1}}}}
\newcommand{\KeywordTok}[1]{\textcolor[rgb]{0.13,0.29,0.53}{\textbf{#1}}}
\newcommand{\NormalTok}[1]{#1}
\newcommand{\OperatorTok}[1]{\textcolor[rgb]{0.81,0.36,0.00}{\textbf{#1}}}
\newcommand{\OtherTok}[1]{\textcolor[rgb]{0.56,0.35,0.01}{#1}}
\newcommand{\PreprocessorTok}[1]{\textcolor[rgb]{0.56,0.35,0.01}{\textit{#1}}}
\newcommand{\RegionMarkerTok}[1]{#1}
\newcommand{\SpecialCharTok}[1]{\textcolor[rgb]{0.00,0.00,0.00}{#1}}
\newcommand{\SpecialStringTok}[1]{\textcolor[rgb]{0.31,0.60,0.02}{#1}}
\newcommand{\StringTok}[1]{\textcolor[rgb]{0.31,0.60,0.02}{#1}}
\newcommand{\VariableTok}[1]{\textcolor[rgb]{0.00,0.00,0.00}{#1}}
\newcommand{\VerbatimStringTok}[1]{\textcolor[rgb]{0.31,0.60,0.02}{#1}}
\newcommand{\WarningTok}[1]{\textcolor[rgb]{0.56,0.35,0.01}{\textbf{\textit{#1}}}}
\usepackage{longtable,booktabs}
% Correct order of tables after \paragraph or \subparagraph
\usepackage{etoolbox}
\makeatletter
\patchcmd\longtable{\par}{\if@noskipsec\mbox{}\fi\par}{}{}
\makeatother
% Allow footnotes in longtable head/foot
\IfFileExists{footnotehyper.sty}{\usepackage{footnotehyper}}{\usepackage{footnote}}
\makesavenoteenv{longtable}
\usepackage{graphicx,grffile}
\makeatletter
\def\maxwidth{\ifdim\Gin@nat@width>\linewidth\linewidth\else\Gin@nat@width\fi}
\def\maxheight{\ifdim\Gin@nat@height>\textheight\textheight\else\Gin@nat@height\fi}
\makeatother
% Scale images if necessary, so that they will not overflow the page
% margins by default, and it is still possible to overwrite the defaults
% using explicit options in \includegraphics[width, height, ...]{}
\setkeys{Gin}{width=\maxwidth,height=\maxheight,keepaspectratio}
% Set default figure placement to htbp
\makeatletter
\def\fps@figure{htbp}
\makeatother
\setlength{\emergencystretch}{3em} % prevent overfull lines
\providecommand{\tightlist}{%
  \setlength{\itemsep}{0pt}\setlength{\parskip}{0pt}}
\setcounter{secnumdepth}{5}

\title{Yet Another Trading Algorithm}
\usepackage{etoolbox}
\makeatletter
\providecommand{\subtitle}[1]{% add subtitle to \maketitle
  \apptocmd{\@title}{\par {\large #1 \par}}{}{}
}
\makeatother
\subtitle{Technical Manual}
\author{Grandez}
\date{2021-05-21}

\begin{document}
\maketitle

{
\setcounter{tocdepth}{1}
\tableofcontents
}
\hypertarget{prerequisites}{%
\chapter{Prerequisites}\label{prerequisites}}

\begin{Shaded}
\begin{Highlighting}[]
\KeywordTok{install.packages}\NormalTok{(}\StringTok{"bookdown"}\NormalTok{)}
\KeywordTok{library}\NormalTok{(abind)}
\KeywordTok{library}\NormalTok{(png)}
\KeywordTok{library}\NormalTok{(grid)}
\KeywordTok{library}\NormalTok{(fontawesome)}

\CommentTok{# or the development version}
\CommentTok{# devtools::install_github("rstudio/bookdown")}
\end{Highlighting}
\end{Shaded}

\hypertarget{introduccion}{%
\chapter*{Introduccion}\label{introduccion}}
\addcontentsline{toc}{chapter}{Introduccion}

XXX Aqui la introduccion

\hypertarget{technical-decissions}{%
\chapter{Technical Decissions}\label{technical-decissions}}

\hypertarget{componentes-web-widgets}{%
\section{\texorpdfstring{Componentes Web (\emph{widgets})}{Componentes Web (widgets)}}\label{componentes-web-widgets}}

Como ya se ha indicado, las páginas Web están basadas en Shiny el cual aparte de definir de manera unívoca el método de programación (basado en los conceptos \emph{reactive} y \emph{bindings}) ofrece un conjunto de \emph{widgets} ya diseñados y listos para usar.

Sobre Shiny se utilizan diferentes paquetes que lo enriquecen de alguna u otra manera, evitando tener que diseñar widgets que ya han sido diseñados y probados, con la limitación de tener que adscribirse a su nomenclatura e implementación.

Normalmente estos paquetes o bien sobrecargan las funciones nativas de Shiny o, lo mas común, utilizan algun criterio de nomenclatura que no colisione con Shiny; por ejemplo, Shiny ofrece una función ``alert'' y el paquete ``\emph{shinywidgets}'' ofrece la misma funcionalidad con una versión con mas opciones que denomina ``shinyAlert''.

\hypertarget{caso-de-uso}{%
\subsection{Caso de uso}\label{caso-de-uso}}

A lo largo del ciclo de vida del sistema es posible que alguno de los paquetes utilizados dejen de existir o de estar soportados, o que se liberen otros paquetes con otras funcionalidades o mejoras.

Por ejemplo, supongamos:

Primero; que disponemos de un paquete que implementa un UI mas atractivo para los botones que el actual, el nuevo paquete denomina a los botones \texttt{prettyButtonxxx} y los actuales se denominan \texttt{simpleButton}

Segundo; que aunque no teníamos las herramientas, por cuestiones de diseño y programación contemplamos diferentes botones, digamos uno para aceptar una operación y otro para anularla. Como no tenemos las herramientas los dos son iguales, pero de manera proactiva hemos creado dos funciones: \texttt{btnOK} y \texttt{btnOK}

En el primer caso, para implementar el nuevo paquete deberíamos analizar todo el sistema en busca de las funciones \texttt{simpleButton}, cambiarlos por el equivalente \texttt{prettyButton} y posiblemente ajustar los parámetros de llamada. Lo cual, según el paquete a integrar podría ser, aparte de arriesgado y costoso, directamente inviable.

Por otro lado, fijémenos en el segundo caso, en este no estamos invocando directamente a \texttt{simpleButton} si no a una función propia que actualmente son iguales, por lo que para cambiar el comportamiento de, digamos \texttt{btnKO} solo deberiamos modificar una función y estaría disponible para todo el sistema.

\hypertarget{decisiuxf3n}{%
\subsection{Decisión}\label{decisiuxf3n}}

El enfoque utilizado para solventar este problema es el descrito en el ejemplo anterior.

En lugar de invocar directamente a los widgets de Shiny o del resto de paquetes, \textbf{todos y cada uno de ellos} están encapsulados en funciones propias del sistema, de manera que la dependencia de este con los paquetes solo existen en un módulo determinado.

El criterio utilizado es el siguiente:

\begin{itemize}
\tightlist
\item
  Los componentes de diseño se denominan \texttt{yuiCamelCaseName}
\item
  Los componentes de servidor se denominan \texttt{updCamelCaseName}
\end{itemize}

Donde, la función \emph{yui} encapsula el widget asociado y la función \emph{upd} gestiona la interacción con ese componente.

De manera habitual, pero no en todos los casos, el nombre del par de funciones se corresponderá con su correspondiente widget o de la funcionalidad que implementa; por ejemplo, si tenemos un boton generico, se llamará \texttt{yuiButton} aunque podemos tener otros mas botones mas especificos como \texttt{yuiButtonOK} o \texttt{yuiButtonKO}

\hypertarget{wrappers}{%
\subsection{Wrappers}\label{wrappers}}

qa1

\hypertarget{about}{%
\section{About}\label{about}}

En esta sección se detallan algunas de las decisiones de diseño, técnicas, etc. que se han tomado, asi como el porqué y sus alternativas.

El objetivo es mantener la trazabilidad y ofrecer un mejor conocimiento del sistema, evitando ``repensar'' de manera repetida la solución a ciertos problemas.

Por facilidad, cada decisión va en su propio archivo

\hypertarget{bloque-de-decisiones-una}{%
\section{Bloque de decisiones una}\label{bloque-de-decisiones-una}}

Aqui se indican las decisiones que se han tomado y por que

\hypertarget{yatadb}{%
\section{YATADB}\label{yatadb}}

This is the package encapsulating the persistence layer

\end{document}
